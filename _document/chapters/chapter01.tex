Refactoring es realizar una  transformación al software preservando su comportamiento, modificando sólo su estructura interna para mejorarlo. El término es de Opdyke, quien lo introdujo por primera vez en 1992, en su tesis doctoral. Fowler menciona que eran cambios realizados en el software para hacerlo más fácil de modificar y comprender, por lo que no son una optimización del código, ya que esto en ocasiones lo hace menos comprensible, ni solucionar errores o mejorar algoritmos. Las refactorizaciones pueden verse como un tipo de mantenimiento preventivo, cuyo objetivo es disminuir la complejidad del software en anticipación a los incrementos de complejidad que los cambios pudieran traer.

Este proyecto de investigación, refactorizará un proyecto desarrollado en CSharp que necesita optimizar su código fuente; además, se resolverá algunos problemas que presentó durante la puesta en producción.

La metodología utilizada en la proyecto, es sencilla de entender y puede servir como guía para aplicarse en proyectos que presenten los mismos bad smells.

\section{Planteamiento del Problema}

El software Matriculas, es un sistema que colabora con la gestión de matrículas para instituciones de educación básica. El sistema fue desarrollado en el año 2016 y cumple con los requisitos especificados por el cliente. sin embargo, presenta problemas de rendimiento y el nivel de mantenibilidad que presenta es inferior de lo esperado.

El cliente requiere que se optimice la arquitectura utilizada en el sistema, para escalarlo sin ningún problema y no se utilice esfuerzos de más en las tareas de implementación.

Los problemas que presenta el sistema, pueden provocar errores que el sistema no pueda controlar, y se requiera parar las operaciones de la institución para resolver los problemas.

\section{Objetivos}

\subsection{General}
Incrementar el nivel de mantenibilidad de la Aplicación Web de Matrículas a través de Refactoring

\subsection{Específicos}
\begin{itemize}
    \item Investigar técnicas de Refactoring
    \item Estudiar la Aplicación Web de Matrículas
    \item Aplicar técnicas de Refactoring 
    \item Validar la solución
\end{itemize}

\section{Justificación}
Es muy importante crear software que sea eficiente y que sea fácil de entender; por lo que es recomendable mantener el código con las mejores prácticas e desarrollo de software. 

El refactoring es una técnica de ingeniería de software que permite restructurar el código fuente sin modificar el comportamiento. Esta técnica nos ayudará a incrementar le nivel de mantenibilidad de la aplicación y tener un código más limpio y óptimo.
